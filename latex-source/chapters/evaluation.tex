Im folgenden Kapitel wurde der Praxistest dokumentiert und die Ergebnisse ausgewertet, um eine Evaluation des prototypisch implementierten Konzepts durchzuführen und daraus entstandene Optimierungsvorschläge zu beschreiben.
\subsection{Praxistest}
Für den Praxistest wurden mit einem Experten, welcher Portfoliomanagement im täglichen Geschäft als Global Key Account Manager verwendet, praxisnahe Daten erstellt, die ein fiktives Unternehmen darstellen sollen. Anhand der Daten im Prototypen soll, beurteilt werden, wie praxisrelevant die Funktionalitäten der Software sind und welche Verbesserungen und/oder Funktionalitäten für eine produktive Nutzung benötigt werden. Das Praxisprojekt beschreibt eine Unternehmensentwicklung für ein Unternehmen. Diese Unternehmensentwicklung stellt KPI's(Key Performance Indicators) als konkrete Unternehmensstrategie-Ziele dar, welche die oberste Ebene im Tool darstellt. Darunter unterteilt sich das Unternehmen in verschiedene Regionen, welche selbst eigene KPI's erfüllen sollen, die gemeinsam auf die KPI's in der obersten Ebene einzahlen. Diese Regionen unterteilen sich erneut in Subregions, welche wieder ihre eigenen KPI's erfüllen sollen. Unter diesen Subregions befindet sich eine HQ-Ebene, in der sich CoE's (Center of Excellence) befinden, welche dann mit den Projekten in der konkreten operativen Ebene darunter verknüpft.

\emph{Dashboard Bild der praxisnahen Daten}

\subsection{Optimierungsvorschläge}
Nach der Erstellung der Datenstruktur hat der Experte das Dashboard zusehen bekommen und konnte sich in der Anwendung ansehen. Dabei wurden bewusste Entscheidungen, wie die fehlende Synchronisation und Autorisierung erläutert und begründet.
Das gegebene Feedback lässt sich in 3 Punkten zusammenfassen:

\begin{itemize}
    \item mehr Filterfunktionalitäten
    \item zeitbasierte Metriken
    \item möglichst vollständige Dokumentation für die dokumentierte Datenstruktur
\end{itemize}

Je größer und komplexer die Datenstruktur wurde, desto schwieriger wurde es die Übersicht über alle Elemente zu behalten. Da meist gar nicht alle Daten in jeden Kontext relevant sind, wäre eine Filterfunktionalität sinnvolle, insbesondere für Planungselemente, in denen noch kein Fortschritt sichtbar ist, da sie sich noch nicht in der Umsetzung befinden.

Fortschritt als aktuellen Wert sichtbar zu machen kann in vielen Situationen bereits bei der Entscheidungsfindung helfen, kann aber auch ohne Kontext von Fortschritt über Zeit irreführend sein. Zeitbasierte Metriken wie den Fortschritt über Zeit sichtbar zu machen, Durchsatz oder die Velocity werden an dieser Stelle ebenfalls benötigt.

Die Datenstruktur sollte so vollständig wie möglich dokumentiert werden können, um effektiver Einigkeit über die Bedeutung der verschiedenen Ebenen und Elemente zu erzielen. Dokumentation sollte dabei direkt im Tool möglich sein.