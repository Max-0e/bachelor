\subsection{Motivation}
Damit Projekte mit hoher Volatilität in ihren Anforderungen effektiv umgesetzt werden können, werden bereits agile Methoden eingesetzt. Diese Methoden werden häufig auch auf höhere Unternehmensebenen skaliert. Damit diese Skalierung erfolgreich sein kann, müssen Metriken für diese Ebenen ermittelt werden, welche eine Aggregation der darunterliegenden Planungselemente sind. Diese Aggregation über verschiedene Ebenen stellt eine Herausforderung dar, wenn in den verschiedenen Ebenen unterschiedliche Lösungen für die Erfassung der Metriken verwendet werden. Hierbei sollen Möglichkeiten der Automatisierung dieser Aggregationen mithilfe einer einheitlichen Softwarelösung untersucht werden. Anschließend wird diese Aggregation prototypisch durch eine solche Softwarelösung implementiert und getestet.
% Kanban ist ein agiles Kommunikations-Framework, welches die Reaktionsfähigkeit und Effizienz eines Projektteams steigern soll. Dies wird durch einen Planungsprozess erreicht, der konstante neue Produktiterationen vorsieht und Arbeitsprozesse von Priorisierung und WIP-Limits abhängig macht.
% Klassisch wird Kanban in agilen Softwareentwicklungsprojekten mit Entwicklungsteams von 8 bis 12 Teammitgliedern angewendet.
% Die Flight-Level Methode beschränkt das Modell Kanban nicht mehr auf Projektteams, sondern sieht Anwendung in allen Unternehmensebenen, auch Flight-Level genannt, vor. Wird diese Methode erfolgreich auf allen Ebenen eingesetzt, erreicht die Organisation den sogenannten Status der Business-Agilität\cite{agilitaetNeuDenken}.
% Die Methode wurde von Klaus Leopold entwickelt und beinhaltet diese drei Flight-Level, die im Weiteren betrachtet werden\cite{agilesProjektmanagementImBerufsalltagFlightLevel}:

\subsection{Zielsetzung}
Das Ziel der Arbeit ist es, ein Konzept zu entwickeln, welches die Aggregation von Metriken über verschiedene Unternehmensebenen hinweg ermöglicht. Dieses Konzept soll anschließend prototypisch implementiert und validiert werden.

\subsection{Methodik}


\subsection{Gliederung der Arbeit}
In dieser Arbeit werden zunächst die grundlegenden Prinzipien des agilen Manifests erläutert und in einen Zusammenhang mit Projekt und Portfoliomanagement gebracht. Anschließenden werden Reporting und wertebasierte Entscheidungsfindung in verschiedenen Unternehmensbereichen erklärt und wie diese in einer gesamten Unternehmensstuktur Anwendung finden. Daraufhin wird ein Konzept erarbeitet, welches in einem Softwareprotoypen abgebildet wird, das die Dokumentation relevanter Informationen beliebig strukturierter Unternehmen erlaubt und somit automatisiertes Reporting wie z.B. Fortschrittsmessung ermöglicht. Ein Praxistest soll zuletzt den implementierten Protoypen evaluieren und eine kritische Reflexion auf das erarbeitete Konzept bieten.