\subsection{Motivation}
Damit Projekte mit hoher Volatilität in ihren Anforderungen effektiv umgesetzt werden können, haben sich bereits agile Methoden etabliert. Diese Methoden werden häufig auch auf höhere Unternehmensebenen skaliert. Damit diese Skalierung erfolgreich sein kann, müssen Metriken für diese Ebenen ermittelt werden, welche eine Aggregation der darunterliegenden Planungselemente darstellen. Diese Aggregation über verschiedene Ebenen stellt eine Herausforderung dar, wenn in den verschiedenen Ebenen unterschiedliche Lösungen für die Erfassung der Metriken verwendet werden.
% Hierbei sollen Möglichkeiten der Automatisierung dieser Aggregationen mithilfe einer einheitlichen Softwarelösung untersucht werden. Anschließend wird diese Aggregation prototypisch durch eine solche Softwarelösung implementiert und getestet.

\subsection{Zielsetzung}
Das Ziel der Arbeit ist es, ein Konzept zu entwickeln, welches die Aggregation von Metriken über verschiedene Unternehmensebenen hinweg ermöglicht. Dieses Konzept soll anschließend prototypisch implementiert und validiert werden.

\subsection{Methodik}


\subsection{Gliederung der Arbeit}
In dieser Arbeit werden zunächst die grundlegenden Prinzipien des agilen Manifests erläutert und in einen Zusammenhang mit Projekt und Portfoliomanagement gebracht. Anschließenden werden Reporting und wertebasierte Entscheidungsfindung in verschiedenen Unternehmensbereichen erklärt und wie diese in einer gesamten Unternehmensstruktur Anwendung finden. Daraufhin wird ein Konzept erarbeitet, welches in einem Software-Protoyp abgebildet wird, das die Dokumentation relevanter Informationen beliebig strukturierter Unternehmen erlaubt und somit automatisiertes Reporting wie z. B. Fortschrittsmessung ermöglicht. Ein Praxistest soll zuletzt den implementierten Prototypen evaluieren und eine kritische Reflexion auf das erarbeitete Konzept bieten.