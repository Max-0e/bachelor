\subsection{Motivation}
Damit Prozessplanungen mit hoher Volatilität in ihren Anforderungen effektiv umgesetzt werden können, haben sich bereits agile Methoden etabliert. Diese Methoden werden häufig auch auf höhere Unternehmensebenen skaliert. Damit diese Skalierung erfolgreich sein kann und auf allen Ebenen wertgetriebene Entscheidungen getroffen werden können, müssen Metriken für diese Ebenen ermittelt werden, welche eine Aggregation der darunterliegenden Planungselemente darstellen. Diese Aggregation über verschiedene Ebenen stellt eine Herausforderung dar, wenn in den verschiedenen Ebenen unterschiedliche Lösungen für die Dokumentation ihrer Arbeit und die Erfassung der Metriken verwendet werden, welche benötigt werden, um die Aggregation durchzuführen.

\subsection{Zielsetzung}
Das Ziel der Arbeit ist die Entwicklung eines Konzepts für die Dokumentation einer Unternehmensstruktur mit der Planungselemente aus verschiedenen Ebenen in einen Zusammenhang gebracht werden können, um über diese Zusammenhänge Metriken aggregieren zu können. Als Beispiel für diese Metriken wird der Fortschritt verwendet. Kern dieses Konzepts ist eine flexible Datenstruktur, welche das Ziel hat möglichst jede Unternehmensstruktur abzubilden, unabhängig von dem verwendeten Struktur-Framework und die Visualisierung diese Datenstruktur. Dieses Konzept wird anschließend prototypisch implementiert und validiert.

\subsection{Methodik}
Durch eine Literaturanalyse Projekt und Projekt-Portfoliomanagement insbesondere im Kontext der Agilität erläutert. Dabei wurde ein Kontext zwischen der Skalierung von agilen Methoden auf höhere Unternehmensebenen geschaffen und dargestellt welche Rolle Metriken und das dafür benötigte Reporting für höhere Unternehmensebenen spielen. Anschließend wurde untersucht, wie wertebasierte Entscheidungsfindung in verschiedenen Unternehmensbereichen funktioniert und wie diese in einer gesamten Unternehmensstruktur Anwendung finden. Zuletzt wurde auf Basis dieser Erkenntnisse ein Konzept für einen webbasierten Prototyp entwickelt und implementiert, welches die Dokumentation relevanter Informationen beliebig strukturierter Unternehmen erlaubt und somit automatisiertes Reporting am Beispiel der Fortschrittsmessung ermöglicht, welcher anschließend evaluiert wurde.

\subsection{Gliederung der Arbeit}
Kapitel 2 und 4 erläutern Projekt und Projekt-Portfoliomanagement vor allem in Kontext der Agilität und wie Reporting und daraus resultierende Metriken in der Entscheidungsfindung in verschiedenen Unternehmensebenen verwendet werden.

Aus diesen Erkenntnissen beschreibt Kapitel 4 die Entwicklung des Konzepts. Hierzu wird ein Prozess erarbeitet, welcher die Dokumentation beschreibt und Anforderungen formuliert, welche durch das Konzept erfüllt werden sollen. Anschließend wird ein UX-Konzept beschrieben, welches die Nutzerinteraktion mit der Datenstruktur in Form einer Weboberfläche abbildet. Zuletzt wird ein Algorithmus für die Aggregation von Metriken am Beispiel des Fortschritts entwickelt.

Kapitel 5 dokumentiert die Implementierung des Konzepts in Form eines Prototyps. Hier wird zunächst die Datenstruktur und die Architektur sowohl des Backends für die Verwaltung dieser Datenstruktur, als auch die Architektur des Frontends für die Nutzerinteraktion und Visualisierung dieser Datenstruktur beschrieben.
Zuletzt wird die Inbetriebnahme des Prototyps erläutert.

Für die Evaluation wird in Kapitel 6 der Praxistest beschrieben, welcher die Anwendung des Prototyps mit praxisnahen Daten simuliert und somit Feedback eines Experten ermöglicht.

Abschließend wird in Kapitel 7 eine kritische Reflexion auf das erarbeitete Konzept und den implementierten Prototypen gegeben und ein Ausblick auf mögliche Weiterentwicklungen auf Basis des Ergebnisses der Evaluation gegeben.
