In dieser Arbeit wurde ein Konzept für eine softwaregestützten Dokumentation von Unternehmensstrukturen für automatisierte Fortschrittsmessung und Werteorientierung entwickelt. Dazu wurde eine Datenstruktur entworfen, die die Abbildung von Unternehmensstrukturen, wie Sie für (agiles) Projekt-Portfoliomanagement verwendet werden und eine Datenaggregation für den Fortschritt ermöglicht. Sie soll dabei möglichst flexibel sein, um unabhängig von der konkreten Unternehmensstruktur bzw. dem verwendeten Management-Framework verwendet werden zu können. Die Datenstruktur wurde in einem Prototyp implementiert, der die Funktionalitäten der Datenstruktur in einer Webanwendung umsetzt, um damit das Konzept in einem Praxistest zu evaluieren. Dazu wurde eine fiktive praxisnahe Unternehmensstruktur mit der Unterstützung eines Experten in der Anwendung abgebildet. Dieses Beispiel wurde dann verwendet, um die Anwendung selbst zu testen.

\subsection{Ergebnis}
Die Ergebnisse des Literatur-Reviews zeigten, dass Unternehmen Entscheidungen durch softwaregestützte Dokumentation und automatisiertes Reporting unterstützen sollten, um Entscheidungsfindung zu systematisieren und Entscheidungsqualität zu steigern. Die Ergebnisse des Praxistests zeigten, dass die Datenstruktur dazu in der Lage ist Unternehmensstrukturen abzubilden und die Fortschrittsaggregation eine Form von automatisiertem Reporting darstellt. Des Weiteren sind zwei Kernfunktionalitäten während der Entwicklung des Konzepts identifiziert worden, welche aufgrund der Komplexität nicht implementiert wurden: Die Synchronisation mit operativen Tools wie Jira, anstelle eines einfachen Imports und die Autorisierung und Verwaltung verschiedener Nutzer für Lesebeschränkungen. Außerdem wurden durch den Praxistest drei weitere Funktionalitäten identifiziert, mit denen die Anwendung möglicherweise produktive Anwendung finden kann.

\subsection{Reflexion}
Das Konzept kann Unternehmensstrukturen als solches dokumentieren, sollte aber weitere Möglichkeiten bieten den Kontext von z. B. verschiedenen Ebenen besser zu beschreiben. Die Fortschrittsaggregation ist eine Möglichkeit für eine Unterstützung von wertebasierter Entscheidungsfindung, reicht aber alleine nicht aus, um effektive die Entscheidungen zu verbessern. Hierzu werden vor allem zeitbasierte Metriken benötigt. Die beiden zuvor erwähnten Funktionalitäten, welche bewusst nicht implementiert wurden, werden definitiv benötigt, um die Anwendung produktiv und Unternehmensweit zu verwenden. Die Visualisierung der Datenstruktur kann ebenfalls verbessert werden, indem Filterfunktionalitäten implementieren, welche die Übersicht über größere Strukturen gewährleisten können und den Fokus auf die relevanten Informationen legen.

Für die Berechnung von zeitbasierten Metriken wird in der Datenstruktur eine zeitliche Dimension benötigt, mit der Änderungen über Zeit gespeichert werden können.

Die Datenaggregation findet zurzeit im Frontend, also dem Nutzer-seitigen Teil der Anwendung, statt. Dies könnte insbesondere bei komplexeren Metriken sinnvoller im Server-seitigen Teil der Anwendung stattfinden. Auch die Kalkulation der Metriken wie Valocity o. Ä. könnte vor allem für zukünftig Synchronisierte Projekt aus anderen Tools, wie z. B. Jira bezogen werden, die ohnehin solche Reports zur Verfügung stellen.

\subsection{Ausblick}
Sowohl das entwickelte Konzept, als auch die Implementierung dessen, erlauben ohne weitere Komplikation eine Umsetzung aller identifizierten fehlenden Funktionalitäten und Verbessungen. Hierbei sollten weitere Metriken, die über die Fortschrittsmessung hinaus gehen, implementiert werden, da es sich bei dieser Arbeit nur um einen Prototypen handelt, der die Aggregation von Daten über mehrere Ebenen hinweg am Beispiel der Fortschrittsmessung demonstriert.
Mit dieser Umsetzung kann das Konzept dann auch realistisch in einem tatsächlichen Praxisumfeld über einen längeren Zeitraum getestet werden, um die Effektivität der Unterstützung von wertebasierter Entscheidungsfindung zu evaluieren.
