\subsection{Reporting in agilen Unternehmen}
Reporting ist der Schlüsselfaktor für den Erfolg der Unternehmensorganisation durch z.B. agiles Portfoliomanagement, da es einen möglichst vollständigen und tiefgehenden Überblick erzeugen kann, auf Basis dessen Entscheidungen getroffen werden. \cite{}
Das Reporting sollte hierbei Einblick in verschiedene Detailgrade der Organisation bieten und somit für jeden Punkt, an dem Entscheidungen getroffen werden,...
\subsection{qualitatives vs. quantitatives Reporting}
Für effektives Reporting müssen verscheidene Metriken erhoben werden, hierbei unterscheided man allgemein in qualitatives und quantitatives Reporting.
Qualitatives Reporting zeigt Chancen auf und bietet Kontext, während quantitatives Reporting das quantifizieren von Elementen und Fortschritt sowie die Validierung von Zielen und geschaffenem Wert ermöglicht \cite{}
\subsection{automatisches Reporting}
Reporting ist meist ein manueller Prozess, welcher mit immer wiederkehrendem Aufwand verbunden ist, da die Metriken regelmäßig erhoben werden müssen. Um das Reporting zu Optimieren, sollten qualitative und quantitative Reports unterschiedlich betrachtet werden. 
Quantitative Metriken sind quantifizierbar, sodass der Prozess der Erhebung dieser Metriken bei vollständiger dokumentation aller relevanter Daten automatisierbar ist. Werden diese Metriken dann automatisch erhoben sorgt dies für konsistentere, regelmäßigere, validere und aktuellere Ergebnissen. 
Qualitative Metriken dagen sind schwer automatisierbar, da sie häufig nicht auf objektiv erfassbaren Daten beruht. Zur Optimierung kann eine systematische Herangehensweise für die Bestimmung der Metriken definiert werden, um mit deren Hilfe  mehr Konsitenz und Regelmäßigkeit zu gewährleisten. Des Weiteren kann man davon ausgehen, dass künstliche Intelligenz in Zukunft eingesetzt werden kann um auch qualitative Metriken weitestgehend zu automatisieren.

\subsection{Reports in Portfoliomanagement}


\subsection{Reports für Value based Software-Engineering}


\subsection{Teamkoordination}

