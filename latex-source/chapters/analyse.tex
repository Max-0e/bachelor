\subsection{Reporting für agiles Portfoliomanagement}
Studien zeigten bereits eine positive Korrelation zwischen erfolgreichem Portfoliomanagement und sogenannter Project Portfolio Control (PPC). PPC wird durch drei Faktoren charakterisiert: Projektauswahl, Reporting und Stil der Entscheidungsfindungs \cite{}.
% Project Portfolio Control and Portfolio Management Performance P.39
% Project Portfolio Control and Portfolio Management Performance P.38
Für die optimale Projektauswahl in PPC wurde bereits untersucht, dass die Entscheidungsfindung optimiert werden kann, indem die Metriken aus dem Reporting, welche für die Entscheidungsfindunge herangezogen werden, mithilfe eines fuzzy Analytic Hierarchy Process (AHP) für eine Priorisierung einzelner Elemente des Portfolios gewichtet werden und anschließen mit der fuzzy TOPSIS Methode in eine Reihenfolge gebracht werden \cite{}.
% The optimal project selection in portfolio management using fuzzy multi-criteria decision- making methodology P.13
Hieraus lässt sich ableiten, dass es keine generalisierbaren Metriken gibt, die für alle Unternehmen und Projekte gelten, sondern dass die Metriken für jedes Unternehmen und jedes Projekt individuell bestimmt werden müssen. Im Optimalfall sollten also grundsätzlich alle bzw. möglichst viele  Metriken erhoben werden, um sie anschließend zu gewichten.

\subsection{qualitatives vs. quantitatives Reporting}
Für effektives Reporting müssen verschiedene Metriken erhoben werden, hierbei unterscheidet man allgemein in qualitatives und quantitatives Reporting.
Qualitatives Reporting zeigt Chancen auf und bietet Kontext, während quantitatives Reporting das Quantifizieren von Elementen und Fortschritt sowie die Validierung von Zielen und geschaffenem Wert ermöglicht \cite{}.
% Settina und Schoemaker P.212

\subsection{Reports für Value based Software-Engineering}
Value-based Software-Engineering(VBSE) ist eine Sammlung von Frameworks für die Entscheidungsfindung in der Softwareentwicklung. VBSE basiert auf der Annahme, dass die Entscheidungen in der Softwareentwicklung auf Basis von einem Kriterium, welches als Wert bezeichnet wird, getroffen werden sollten \cite{}.
Wert kann hier auf verschiedene Arten definiert werden und kann in mehrere Teile heruntergebrochen werden. Beispiele für Wert sind:
\begin{itemize}
  \item Nutzen
  \item Kosten
  \item Risiko
  \item Wert für den Kunden
  \item Wert für das Unternehmen.
\end{itemize}

Unter der Berücksichtigung des Werts können Entscheidungen wertezentiert getroffen werden. Diese Entscheidungen verteilen sich über den gesamten Software-Engineering-Prozess, welcher auch  als VBSE Agenda bezeichnet wird \cite{}.
Der Prozess kann folgende Teile beinhalten:
\begin{itemize}
  \item Requirements Engineering
  \item Architecting
  \item Design und Entwicklung
  \item Verifizierung und Validation
  \item Planung und Kontrolle
  \item Risikomanagement
  \item Qualitätsmanagement
  \item Mitarbeitermanagement
\end{itemize}

\subsection{Teamkoordination}
Teamkoordination ist ein wichtiger Bestandteil in jeder Form von Projektplanung und -durchführung. In agilen Unternehmen ist Teamkoordination besonders wichtig, da die Teams selbstorganisiert sind und somit die Koordination der Teams untereinander nicht von einer zentralen Instanz übernommen wird und ebenfalls die Projektverantwortung in das Team gegeben wird \cite{}.
Somit wird eine gute Teamkoordination kritisch für den Erfolg des Projekts \cite{}.

\subsection{Automatisches Reporting}
Reporting ist meist ein manueller Prozess, welcher mit immer wiederkehrendem Aufwand verbunden ist, da die Metriken regelmäßig erhoben werden müssen. Um diesen Prozess des Reportings zu optimieren, sollten qualitative und quantitative Reports unterschiedlich betrachtet werden.
Quantitative Metriken sind quantifizierbar, sodass der Prozess der Erhebung dieser Metriken bei vollständiger Dokumentation aller relevanter Daten automatisierbar ist. Werden diese Metriken dann automatisch erhoben sorgt dies für konsistente, regelmäßige, valide und aktuelle Ergebnissen \cite{}.

Qualitative Metriken dagegen sind schwer automatisierbar, da sie häufig nicht auf objektiv erfassbaren Daten beruht. Zur Optimierung kann eine systematische Herangehensweise für die Bestimmung der Metriken definiert werden, um mit deren Hilfe  mehr Konsistenz und Regelmäßigkeit zu gewährleisten. Des Weiteren kann man davon ausgehen, dass künstliche Intelligenz in Zukunft eingesetzt werden kann, um auch qualitative Metriken weitestgehend zu automatisieren \cite{}.

\subsection{Reporting mit digitalen Tools}

\subsection{Bedeutung für diese Arbeit}
In den vorherigen Kapiteln wurde beleuchtet, dass Agilität skaliert auf Unternehmensebene verschiedene Vorteile und  Risiken mit sich bringt und auf regelmäßige ad hoc Entscheidungen in verschiedenen Ebenen des Unternehmens angewiesen ist, um effektive Ergebnisse zu erzielen. Diese Entscheidungen sollten auf Basis von Metriken aus generalisierten Reporting-Prozessen getroffen werden, wobei diese Reporting-Prozesse, wenn möglich, automatisiert werden sollten, um Konsistenz, Regelmäßigkeit, Validität und Aktualität zu gewährleisten. Tool basierendes Reporting hat sich in vielen Unternehmen als eine erfolgreiche Automatisierungsmethode bewiesen.
In dieser Arbeit soll nun prototypisch ein Tool entwickelt werden, welches die Aggregation von Metriken über mehrere Bestandteile eines Planungselements ermöglicht und somit die Entscheidungsfindung in agilen Unternehmen unterstützt. Hierbei soll nicht nur ein aggregierter Wert ermittelt werden, sondern auch der Entstehungsweg des Werts visuell transparent gemacht werden.
