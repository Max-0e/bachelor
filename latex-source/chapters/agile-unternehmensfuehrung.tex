\subsection{Warum agil?}
Agilität im Kontext von Projektmanagement oder auch grundsätzlicher Unternehmensorganisation ist ein alternativer Ansatz für die Planung unternehmensinterner Prozesse, wie z. B. die Umsetzung eines Projekts. Dieser alternative Ansatz entstand durch den Bedarf Projekte effektiv zu managen, die zunehmend komplexer und unsicherer in ihrem Verlauf wurden, sodass ein klassisches Wasserfallmodel große Risiken mit sich zog. Daraus folgend musste eine Methode entwickelt werden, die während der Umsetzung Raum für Anpassungen lässt und die hohe Volatilität dieser Projekte zu verringern. \cite{agilismVsTranditionalApproaches}

Hierbei muss betont werden, dass der agile Ansatz nicht immer der Beste ist. Wie bereits erwähnt entstand er aus dem Bedarf für das Management von komplexen und dynamischen Projekten, bei denen das statische Wasserfallmodel nicht die nötige Flexibilität bieten kann. Hat ein Projekt keinen Bedarf für agile Vorgehensweisen und wird dennoch agil durchgeführt, kann dies zu Verminderung des Projekterfolgs führen. Hierzu wurden bereits Projekte mit verschiedenen Projektmanagementmethoden durchgeführt und der Projekterfolg gemessen. Hierbei zeigte sich, dass die ein hybrider Ansatz zwischen traditionellem und agilem Projekt Management die besten Projekterfolge erzielt. \cite{traditionalAndAgileOnProjectSuccess}
\subsection{Agiles Portfoliomanagement}
Wie zuvor bereits beschrieben, hat der agile Ansatz das Ziel z.B. ein Projekt dynamischer und reaktionsfähiger auf änderung innerhalb des Verlaufs der Umsetzung zu machen. Diese Ansatz kann ebenfalls für das Portfoliomanagement innerhalb eine Unternehmens verwendet werden. Traditionelles Portfoliomanagement oder auch Projekt Portfoliomanagement basiert auf ... \cite{}

Agiles Portfoliomanagement dagegen hat das Ziel Unternehmensziele mit Initiativen oder Projekten zu verknüpfen und somit den Fluss von geleisteter Arbeit auf operativer Ebene zu steuern um diese Ziele zu erreichen und dabei die Dynamik agiler Frameworks beizubehalten. \cite{}

In a first cross-case study comparing the application of agile portfolio man- agement in 14 large organisations to existing literature and professional frame- works, Stettina and Hörz [1] point at the characteristics of agile portfolio manage- ment as (1) transparency of resources and work items, improving trust, decision- making, and resource allocation; (2) collaboration, close collaboration based on routinised interaction and artefacts enabling frequent feedback-loops across the domains; (3) commitment to strategically managed portfolios; (4) team orientation, removing unrest in resource allocation and building capabilities in teams.

\subsection{Agile Unternehmensstrukturierung}

\subsection{Agile Projektstruktur}

\subsection{Beispiel Flight-Level}
