\subsection{Warum agil?}
Agilität im Kontext von Projektmanagement oder auch grundsätzlicher Unternehmensorganisation ist ein alternativer Ansatz für die Planung unternehmensinterner Prozesse, wie z. B. die Umsetzung eines Projekts und steht meist dem sogenannten traditionellen Ansatz gegenüber. Unter diesem traditionellen Ansatz wird für gewöhnlich der lineare Planungsprozess verstanden, welcher voraussetzt, dass Anforderungen vor der Umsetzung klar definiert und dokumentiert sind und somit Risiko minimiert wird. Diese Umstände sind allerdings nicht immer gegeben, bevor ein geplanter Prozess beginnen muss, damit Konkurrenzfähigkeit für ein Unternehmen gegeben ist. Solche zeitkritischen Prozesse sind häufig aber maßgebend für den Erfolg eines Unternehmens, wodurch ein Bedarf für eine Methode entstand, die Anpassbarkeit an sich ändernde Anforderungen und Rahmenbedingungen erlaubt \cite{agilismVsTranditionalApproaches}.

Bei traditioneller Planung erhöht sich durch diese Bedingungen das Risiko die falschen Dinge zum falschen Zeitpunkt zu tun. Agile Methodik erlaubt es diese Prozesse so effektiv wie möglich zu managen, da die Planung nicht linear, sondern iterativ stattfinde. Durch regelmäßige Feedbackschleifen mit Stakeholdern bleibt der Fokus auf Werteorientierung, da sich ändernde Anforderungen regelmäßig in den Planungsprozess der nächsten Iteration einbezogen werden. \emph{Außerdem wird die Projektverantwortung von der Rolle des Projektmanagers ins Team gegeben.} Somit entsteht eine Flexibilität und Anpassbarkeit, welche die hohe Volatilität verringert. Dadurch, dass Dinge erst dann entschieden werden, wenn es notwendig ist, ist allerdings der Gesamtaufwand und die -dauer nicht zu Beginn einschätzbar, sondern immer nur der Aufwand und die Dauer der aktuellen Iteration \cite{agilismVsTranditionalApproaches}.

Ziel bei der Wahl der Planungsmethode ist immer den Erfolg der Umsetzung des geplanten Prozesses zu maximieren. Für Projekte wird dieser Erfolg in zwei Schlüsselfaktoren unterteilt. Kurzfristiger Projekterfolg wird durch die Effizienz definiert, langfristiger Projekterfolg durch Effektivität. Diese beiden Faktoren werden durch Eingrenzung des Projektumfangs, schneller Lieferung, Qualitätssicherung, Kundenzufriedenheit und klare Kommunikation an und zwischen Stakeholdern \cite{traditionalAndAgileOnProjectSuccess}. Im Beispiel des Projektmanagements wurde bereits untersucht, inwiefern die Verwendung von agilen Methoden, den Projekt erfolgt steigert. Dabei stellte sich heraus, dass gerade der Erfolg agiler Projektplanung von der Qualität der Teamarbeit abhängig ist.
Außerdem zeigte sich, dass in den meisten Fällen ein hybrider Ansatz sowohl dem traditionellen als auch dem strikt agilen Ansatz überlegen ist \cite{traditionalAndAgileOnProjectSuccess}.
Hat ein Projekt keinen Bedarf für agile Vorgehensweisen und wird dennoch agil durchgeführt, kann dies zu Verminderung des Projekterfolgs führen. 

\subsection{Agiles Portfoliomanagement}
Wie zuvor bereits beschrieben, hat der agile Ansatz das Ziel z. B. ein Projekt dynamischer und reaktionsfähiger auf Änderung innerhalb des Verlaufs der Umsetzung zu machen. Dieser Ansatz kann ebenfalls für das Portfoliomanagement innerhalb eines Unternehmens verwendet werden. Traditionelles Portfoliomanagement oder auch Projekt Portfoliomanagement basiert auf … \cite{}.

Agiles Portfoliomanagement dagegen hat das Ziel Unternehmensziele mit Initiativen oder Projekten zu verknüpfen und somit den Fluss von geleisteter Arbeit auf operativer Ebene zu steuern, um diese Ziele zu erreichen und dabei die Dynamik agiler Frameworks beizubehalten. \cite{}

% In a first cross-case study comparing the application of agile portfolio man- agement in 14 large organisations to existing literature and professional frame- works, Stettina and Hörz [1] point at the characteristics of agile portfolio manage- ment as (1) transparency of resources and work items, improving trust, decision- making, and resource allocation; (2) collaboration, close collaboration based on routinised interaction and artefacts enabling frequent feedback-loops across the domains; (3) commitment to strategically managed portfolios; (4) team orientation, removing unrest in resource allocation and building capabilities in teams.

\subsection{Agile Unternehmensstrukturierung}

\subsection{Agile Projektstruktur}

\subsection{Beispiel Flight-Level}
