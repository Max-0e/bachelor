\subsection{Warum agil?}
Agilität im Kontext von Projektmanagement oder auch grundsätzlicher Unternehmensorganisation ist ein alternativer Ansatz für die Planung unternehmensinterner Prozesse, wie z. B. die Umsetzung eines Projekts. Dieser alternative Ansatz entstand durch den Bedarf Projekte effektiv zu managen, die zunehmend komplexer und unsicherer in ihrem Verlauf wurden, sodass ein klassisches Wasserfallmodel große Risiken mit sich zog. Daraus folgend musste eine Methode entwickelt werden, die während der Umsetzung Raum für Anpassungen lässt und die hohe Volatilität dieser Projekte zu verringern. \cite{agilismVsTranditionalApproaches}

Hierbei muss betont werden, dass der agile Ansatz nicht immer der Beste ist. Wie bereits erwähnt entstand er aus dem Bedarf für das Management von komplexen und dynamischen Projekten, bei denen das statische Wasserfallmodel nicht die nötige Flexibilität bieten kann. Hat ein Projekt keinen Bedarf für agile Vorgehensweisen und wird dennoch agil durchgeführt, kann dies zu Verminderung des Projekterfolgs führen. Hierzu wurden bereits Projekte mit verschiedenen Projektmanagementmethoden durchgeführt und der Projekterfolg gemessen. Hierbei zeigte sich, dass die ein hybrider Ansatz zwischen traditionellem und agilem Projekt Management die besten Projekterfolge erzielt. \cite{traditionalAndAgileOnProjectSuccess}
\subsection{Agiles Portfoliomanagement}

\subsection{Agile Unternehmensstrukturierung}

\subsection{Agile Projektstruktur}

\subsection{Beispiel Flight-Level}
