\documentclass[12pt]{article}
\usepackage[font=small,labelfont=bf,hypcap=false]{caption}
\usepackage[a4paper,top=20mm,bottom=20mm,right=20mm,left=40mm]{geometry}
% \usepackage[a4paper,top=20mm,bottom=20mm,right=20mm,left=40mm,twoside]{geometry}
% \let\tmp\oddsidemargin
% \let\oddsidemargin\evensidemargin
% \let\evensidemargin\tmp
% \reversemarginpar
\usepackage[utf8]{inputenc}
\usepackage{csquotes}
\usepackage[ngerman]{babel}
\addto{\captionsngerman}{\renewcommand{\abstractname}{Abstract}}
\renewcommand{\baselinestretch}{1.5}
\usepackage{blindtext}
\usepackage{tabularx}
\usepackage{ragged2e}
\usepackage{setspace}
\newcommand\hr{\par\vspace{-.5\ht\strutbox}\noindent\hrulefill\par}
\usepackage{longtable}
\usepackage{hyperref}
\usepackage{listings}
\usepackage{xcolor}
\renewcommand*{\lstlistlistingname}{Codeverzeichnis}

\definecolor{codegreen}{rgb}{0,0.6,0}
\definecolor{codegray}{rgb}{0.5,0.5,0.5}
\definecolor{codepurple}{rgb}{0.58,0,0.82}
\definecolor{backcolour}{rgb}{0.95,0.95,0.92}

\lstdefinestyle{mystyle}{
    backgroundcolor=\color{backcolour},
    commentstyle=\color{codegreen},
    keywordstyle=\color{magenta},
    numberstyle=\tiny\color{codegray},
    stringstyle=\color{codepurple},
    basicstyle=\ttfamily\footnotesize,
    breakatwhitespace=false,
    breaklines=true,
    captionpos=b,
    keepspaces=true,
    numbers=left,
    numbersep=5pt,
    showspaces=false,
    showstringspaces=false,
    showtabs=false,
    tabsize=2
}
\lstset{style=mystyle, language=[Sharp]C}

\lstdefinelanguage{JavaScript}{
  keywords={typeof, new, true, false, catch, function, return, null, catch, switch, var, if, in, while, do, else, case, break},
  keywordstyle=\color{blue}\bfseries,
  ndkeywords={class, export, boolean, throw, implements, import, this},
  ndkeywordstyle=\color{darkgray}\bfseries,
  identifierstyle=\color{black},
  sensitive=false,
  comment=[l]{//},
  morecomment=[s]{/*}{*/},
  commentstyle=\color{purple}\ttfamily,
  stringstyle=\color{red}\ttfamily,
  morestring=[b]',
  morestring=[b]"
}

\lstset{
   language=JavaScript,
   extendedchars=true,
   basicstyle=\footnotesize\ttfamily,
   showstringspaces=false,
   showspaces=false,
   numbers=left,
   numberstyle=\footnotesize,
   numbersep=9pt,
   tabsize=4,
   breaklines=true,
   showtabs=false,
   captionpos=b
}

\lstdefinestyle{mystyle}{
    backgroundcolor=\color{backcolour},   
    commentstyle=\color{codegreen},
    keywordstyle=\color{magenta},
    numberstyle=\tiny\color{codegray},
    stringstyle=\color{codepurple},
    basicstyle=\ttfamily\footnotesize,
    breakatwhitespace=false,         
    breaklines=true,                 
    captionpos=b,                    
    keepspaces=true,                 
    numbers=left,                    
    numbersep=5pt,                  
    showspaces=false,                
    showstringspaces=false,
    showtabs=false,                  
    tabsize=2
}

\lstset{style=mystyle}

\usepackage[
natbib=true,
style=numeric,
sorting=none
]{biblatex}
\addbibresource{references.bib}

\usepackage{fancyhdr}
\setlength{\headheight}{14.5pt}
\renewcommand{\headrulewidth}{0pt}
\fancyfoot{}
\fancyhead{}

\usepackage{graphicx}
\graphicspath{{./images}}

\begin{document}
\pagenumbering{gobble}
\begin{titlepage}
    \begin{center}
        % \begin{flushright}
        %     \includegraphics[width=0.4\textwidth]{universityLogo}\\
        % \end{flushright}
        \vspace{0.5cm}
        \Large{
            \textbf{Hochschule Rhein-Waal}\\
            Fakultät: Kommunikation und Umwelt
        }

        \vspace{3cm}
        \vfill
        \LARGE{
            \textbf{Konzeption und Entwicklung eines Systems zur softwaregestützte Dokumentation von Unternehmensstrukturen für automatisierte Fortschrittsmessung und Werteorientierung 
        }}
        \vfill

        \vspace{1.5cm}
        \LARGE{Bachelorarbeit}
        
        \vfill

        \vspace{1cm}
        \large{vorgelegt von \\}
        \LARGE{Maximilian Oedinger}
    \end{center}
\end{titlepage}
\fancyhead[C]{\thepage}
\pagestyle{fancy}
\pagenumbering{roman}
\newpage
\thispagestyle{empty}
\begin{titlepage}
    \begin{center}
        \begin{flushright}
        \includegraphics[width=0.4\textwidth]{universityLogo}\\
        
        \end{flushright}
        \vspace{3cm}
        \LARGE
        Bachelor-Arbeit\\
        \vspace{1.5cm}
        \LARGE
        
        \textbf{Konzeption und Entwicklung eines Systems für eine softwaregestützte Dokumentation von Flight-Level-Projekten für automatisierte Fortschrittsmessung und Werteorientierung 
    }\\
    
        \vspace{1cm}
        \LARGE
        Maximilian Oedinger
        
        \vfill
        \vspace{0.5cm}
        \Large
        Fakultät Kommunikation und Umwelt\\
        März 2022
        
    \end{center}
    \end{titlepage}
\newpage
\begin{abstract}
    Agility plays a big role in managing the complexity of volatile and unpredictable environments, such as project management and project portfolio management, by focussing on the value created for the customer. This leads to the need of systematically scaling agile practices to enterprise level, which can be done by using e.g. the Scaled Agile Framework (SAFe) or Flight Levels and results in big organizational structures with multiple levels of planning and decision making, which have to be documented.
    \newline
    \newline
    For effective decision making in those environments, decisions should be made with metrics in mind that are relevant for the Level in which the decision must be made. This work presents a customizable software solution to document the organizational structure in a way that allows automatic aggregation of such metrics to the level required. These Metrics should be presented in a meaningful manner, providing insights on progress from levels below the one where the decision must be made in and what implications the decision will have on the levels above. Focus of this work is the adaptability of the solution regardless of the framework used to scale agile practices.
    \newline
    \newline
    The solution was evaluated in a case study to determine its usefulness as a webbased software. The results show that the adaptability of the solution is given, but to be useful in a real world scenario, the solution needs to be extended with time-based metrics like velocity, etc. and the ability to not only import data from operational tools like Jira, but also a synchronization of data between the tools, to make the solution usable over time, since the data only represents a snapshot of the current state of the organization.
    \newline
    \newline
    \textbf{Keywords:} agile portfolio management, agile project management, progress tracking, decision making, Flight Levels, Scaled Agile Framework, data aggregation, visualization, web development, mevn stack
\end{abstract}

\newpage
\tableofcontents
% \newpage
% \addcontentsline{toc}{section}{Abkürzungsverzeichnis}
\section*{Abkürzungsverzeichnis}
\begin{longtable}{p{3cm}p{12cm}}
    Abkürzung & Erklärung \cr
\end{longtable}
% \newpage
% \addcontentsline{toc}{section}{Symbolverzeichnis}
\section*{Symbolverzeichnis}
\begin{longtable}{p{3cm}p{12cm}}
    Symbol & Erklärung \cr
\end{longtable}
\newpage
\addcontentsline{toc}{section}{Abbildungsverzeichnis}
\listoffigures
\addcontentsline{toc}{section}{Codeverzeichnis}
\lstlistoflistings
\newpage
% dis is probably not da way
% \addcontentsline{toc}{section}{Tabellenverzeichnis}
% \section*{Tabellenverzeichnis}

\newpage
\pagenumbering{arabic}

\section{Einleitung}
\subsection{Motivation}
Damit Prozessplanungen mit hoher Volatilität in ihren Anforderungen effektiv umgesetzt werden können, haben sich bereits agile Methoden etabliert. Diese Methoden werden häufig auch auf höhere Unternehmensebenen skaliert. Damit diese Skalierung erfolgreich sein kann und auf allen Ebenen wertgetriebene Entscheidungen getroffen werden können, müssen Metriken für diese Ebenen ermittelt werden, welche eine Aggregation der darunterliegenden Planungselemente darstellen. Diese Aggregation über verschiedene Ebenen stellt eine Herausforderung dar, wenn in den verschiedenen Ebenen unterschiedliche Lösungen für die Dokumentation der Arbeit und die Erfassung dieser Metriken verwendet werden.

\subsection{Zielsetzung}
Das Ziel der Arbeit ist die Entwicklung eines Konzepts für die Dokumentation einer Unternehmensstruktur mit der Planungselemente aus verschiedenen Ebenen in einen Zusammenhang gebracht werden können, damit Metriken über diese Zusammenhänge aggregiert werden können. Als Beispiel für diese Metriken wird der Fortschritt verwendet. Kern dieses Konzepts ist eine flexible Datenstruktur, welche das Ziel hat möglichst jede Unternehmensstruktur abzubilden, unabhängig von dem verwendeten Struktur-Framework und die Visualisierung dieser Datenstruktur. Dieses Konzept wird anschließend prototypisch implementiert und validiert.

\subsection{Methodik}
Durch eine Literaturanalyse wurden Projekt-Management und Projekt-Portfolio-management insbesondere im Kontext der Agilität erläutert. Dabei wurde ein Kontext zwischen der Skalierung von agilen Methoden auf höhere Unternehmensebenen und der Rolle von Metriken und das dafür benötigte Reporting für höhere Unternehmensebenen geschaffen. Anschließend wurde untersucht, wie wertebasierte Entscheidungsfindung in verschiedenen Unternehmensbereichen funktioniert und wie diese in einer gesamten Unternehmensstruktur Anwendung finden. Zuletzt wurde auf Basis dieser Erkenntnisse ein Konzept für einen webbasierten Prototyp entwickelt und implementiert, welches die Dokumentation relevanter Informationen beliebig strukturierter Unternehmen erlaubt und somit automatisiertes Reporting am Beispiel der Fortschrittsmessung ermöglicht, welcher anschließend evaluiert wurde.

\subsection{Gliederung der Arbeit}
Kapitel 2 und 3 erläutern Projekt-Management und Projekt-Portfoliomanagement vor allem im Kontext der Agilität. Zudem wird erklärt wie Reporting und daraus resultierende Metriken in der Entscheidungsfindung in verschiedenen Unternehmensebenen verwendet werden.

Aus diesen Erkenntnissen beschreibt Kapitel 4 die Entwicklung des Konzepts. Hierzu wird ein Prozess erarbeitet, welcher die Dokumentation beschreibt und Anforderungen formuliert, welche durch das Konzept erfüllt werden sollen. Anschließend wird ein UX-Konzept beschrieben, welches die Nutzerinteraktion mit der Datenstruktur in Form einer Weboberfläche abbildet. Zuletzt wird ein Algorithmus für die Aggregation von Metriken am Beispiel des Fortschritts entwickelt.

Kapitel 5 dokumentiert die Implementierung des Konzepts in Form eines Prototyps. Hier wird zunächst die Datenstruktur und die Architektur sowohl des Backends für die Verwaltung dieser Datenstruktur, als auch die Architektur des Frontends für die Nutzerinteraktion und Visualisierung dieser Datenstruktur beschrieben.
Zuletzt wird die Inbetriebnahme des Prototyps erläutert.

Für die Evaluation wird in Kapitel 6 der Praxistest beschrieben, welcher die Anwendung des Prototyps mit praxisnahen Daten simuliert und somit Feedback eines Experten ermöglicht.

Abschließend wird in Kapitel 7 eine kritische Reflexion auf das erarbeitete Konzept und den implementierten Prototypen gegeben und ein Ausblick auf mögliche Weiterentwicklungen auf Basis des Ergebnisses der Evaluation gegeben.

\newpage
\section{Agile Unternehmensführung}
\subsection{Warum agil?}
Agilität im Kontext von Projektmanagement oder auch grundsätzlicher Unternehmensorganisation ist ein alternativer Ansatz für die Planung unternehmensinterner Prozesse, wie z. B. die Umsetzung eines Projekts. Dieser alternative Ansatz entstand durch den Bedarf Projekte effektiv zu managen, die zunehmend komplexer und unsicherer in ihrem Verlauf wurden, sodass ein klassisches Wasserfallmodel große Risiken mit sich zog. Daraus folgend musste eine Methode entwickelt werden, die während der Umsetzung Raum für Anpassungen lässt und die hohe Volatilität dieser Projekte zu verringern. \cite{agilismVsTranditionalApproaches}

Hierbei muss betont werden, dass der agile Ansatz nicht immer der Beste ist. Wie bereits erwähnt entstand er aus dem Bedarf für das Management von komplexen und dynamischen Projekten, bei denen das statische Wasserfallmodel nicht die nötige Flexibilität bieten kann. Hat ein Projekt keinen Bedarf für agile Vorgehensweisen und wird dennoch agil durchgeführt, kann dies zu Verminderung des Projekterfolgs führen. Hierzu wurden bereits Projekte mit verschiedenen Projektmanagementmethoden durchgeführt und der Projekterfolg gemessen. Hierbei zeigte sich, dass die ein hybrider Ansatz zwischen traditionellem und agilem Projekt Management die besten Projekterfolge erzielt. \cite{traditionalAndAgileOnProjectSuccess}
\subsection{Agiles Portfoliomanagement}

\subsection{Agile Unternehmensstrukturierung}

\subsection{Agile Projektstruktur}

\subsection{Beispiel Flight-Level}

\newpage
\section{Analyse}
Im vorherigen Kapitel wurde bereits untersucht wie Projekt- und Projekt-Portfolio-management insbesondere in Kombination mit agiler Methodik die Erreichung strategischer Unternehmensziele unterstützen kann. Durch die regelmäßig anstehenden Entscheidungen besteht der Bedarf für regelmäßiges und vollständiges Reporting, welches für die Entscheidungen herangezogen werden kann, um diese zu objektivieren. In diesem Kapitel soll insbesondere die Automatisierung quantitativer Metriken mit Tools untersucht werden, um Schlüsse für die Umsetzung der Fortschrittsmessung in dieser Arbeit zu ziehen.

\subsection{Reporting für agiles Portfoliomanagement}
Studien zeigten bereits eine positive Korrelation zwischen erfolgreichem Portfoliomanagement und sogenannter Project Portfolio Control (PPC). PPC wird durch drei Faktoren charakterisiert: Projektauswahl, Reporting und Stil der Entscheidungsfindung \cite{ProjectPortfolioControl}.
% Project Portfolio Control and Portfolio Management Performance P.39
% Project Portfolio Control and Portfolio Management Performance P.38
Für die optimale Projektauswahl in PPC wurde bereits untersucht, dass die Entscheidungsfindung optimiert werden kann, indem die Metriken aus dem Reporting, welche für die Entscheidungsfindunge herangezogen werden, mithilfe eines fuzzy Analytic Hierarchy Process (AHP) für eine Priorisierung einzelner Elemente des Portfolios gewichtet werden und anschließen mit der fuzzy TOPSIS Methode in eine Reihenfolge gebracht werden \cite{Mohammed2021}.
% The optimal project selection in portfolio management using fuzzy multi-criteria decision- making methodology P.13
Hieraus lässt sich ableiten, dass es keine generalisierbaren Metriken gibt, die für alle Unternehmen und Projekte gelten, sondern dass die Metriken für jedes Unternehmen und jedes Projekt individuell bestimmt werden müssen. Im Optimalfall sollten also grundsätzlich alle bzw. möglichst viele  Metriken erhoben werden, um sie anschließend zu gewichten.

\subsection{qualitatives vs. quantitatives Reporting}
C. J. Settina und L. Schoemaker \cite{reportingInAgilePortfoliomanagement} untersuchten bereits das Reporting für agiles Portfoliomanagement in einer Fallstudie in mehreren Unternehmen. Aus der Befragung der teilnehmenden Unternehmen ergaben sich folgende Metriken, welche für das Reporting erhoben wurden:

\vspace{20pt}
\begin{center}
  \begin{minipage}{1\linewidth}
    \includegraphics[width=\linewidth]{TableSettinaSchoemaker.png}
    \captionof{figure}{qualitatives und quantitatives Reporting \cite{reportingInAgilePortfoliomanagement} }
  \end{minipage}
\end{center}
\vspace{20pt}


Sie stellten fest, dass für effektives Reporting verschiedene Metriken erhoben werden müssen, welche allgemein in qualitativem und quantitativem Reporting unterteilt werden können.
Qualitatives Reporting zeigt Chancen auf und bietet Kontext, während quantitatives Reporting das Quantifizieren von Elementen und Fortschritt sowie die Validierung von Zielen und geschaffenem Wert ermöglicht.
% Settina und Schoemaker P.212
Aus den gegebenen Metriken kann abgeleitet werden, dass Fortschritt als quantitative Metrik eingeordnet werden kann, da sie vergleichbar mit z. B. Sprint-Burndowns sind, welche letztendlich den Fortschritt über Zeit widerspiegeln.

% \subsection{Reports für Value based Software-Engineering}
% Value-based Software-Engineering(VBSE) ist eine Sammlung von Frameworks für die Entscheidungsfindung in der Softwareentwicklung. VBSE basiert auf der Annahme, dass die Entscheidungen in der Softwareentwicklung auf Basis von einem Kriterium, welches als Wert bezeichnet wird, getroffen werden sollten \cite{}.
% Wert kann hier auf verschiedene Arten definiert werden und kann in mehrere Teile heruntergebrochen werden. Beispiele für Wert sind:
% \begin{itemize}
%   \item Nutzen
%   \item Kosten
%   \item Risiko
%   \item Wert für den Kunden
%   \item Wert für das Unternehmen.
% \end{itemize}

% Unter der Berücksichtigung des Werts können Entscheidungen wertezentiert getroffen werden. Diese Entscheidungen verteilen sich über den gesamten Software-Engineering-Prozess, welcher auch  als VBSE Agenda bezeichnet wird \cite{}.
% Der Prozess kann folgende Teile beinhalten:
% \begin{itemize}
%   \item Requirements Engineering
%   \item Architecting
%   \item Design und Entwicklung
%   \item Verifizierung und Validation
%   \item Planung und Kontrolle
%   \item Risikomanagement
%   \item Qualitätsmanagement
%   \item Mitarbeitermanagement
% \end{itemize}

% \subsection{Teamkoordination}
% Teamkoordination ist ein wichtiger Bestandteil in jeder Form von Projektplanung und -durchführung. In agilen Unternehmen ist Teamkoordination besonders wichtig, da die Teams selbstorganisiert sind und somit die Koordination der Teams untereinander nicht von einer zentralen Instanz übernommen wird und ebenfalls die Projektverantwortung in das Team gegeben wird \cite{}34.
% Somit wird eine gute Teamkoordination kritisch für den Erfolg des Projekts \cite{}.

\subsection{Automatisches Reporting}
Ein weiterer Schluss aus der Arbeit von C. J. Settina und L. Schoemaker \cite{reportingInAgilePortfoliomanagement} ist, dass es sich beim Reporting meist um einen manuellen Prozess handelt, welcher mit wiederkehrendem Aufwand verbunden ist, da die Metriken regelmäßig erhoben werden müssen. Um diesen Prozess des Reportings zu optimieren, sollten qualitative und quantitative Reports unterschiedlich betrachtet werden.

Quantitative Metriken sind quantifizierbar, sodass der Prozess der Erhebung dieser Metriken bei vollständiger Dokumentation aller relevanter Daten automatisierbar ist. Werden diese Metriken dann automatisch erhoben sorgt dies für konsistente, regelmäßige, valide und aktuelle Ergebnissen.

Qualitative Metriken sind dagegen schwer automatisierbar, da sie häufig nicht auf objektiv erfassbaren Daten beruhen. Zur Optimierung kann eine systematische Herangehensweise für die Bestimmung der Metriken definiert werden, um mit deren Hilfe  mehr Konsistenz und Regelmäßigkeit zu gewährleisten. Des Weiteren stellen Sie in ihrer Arbeit die Hypothese auf, dass künstliche Intelligenz in Zukunft eingesetzt werden kann, um auch qualitative Metriken weitestgehend zu automatisieren.
% Settina und Schoemaker P.213

\subsection{Reporting mit digitalen Tools}
Automatisiertes Reporting setzt einen Tool-getriebenen Planungsprozess voraus, um Daten in einem verarbeitbaren Zustand für die Automatisierung zur Verfügung stellen zu können.

\subsubsection{Verwendung von digitalen Tools}
Eine Fallstudie, die mit mehreren IT-Unternehmen durchgeführt wurde, welche aktiv Projekt-Portfoliomanagement betreiben, gibt Empfehlung für eine erfolgreiche Implementierung von Projekt-Portfoliomanagement \cite{guidelinesForPortfoliomanagement}.
Eine dieser Empfehlungen ist die Verwendung eines Systems zur einheitlichen und aktuellen Dokumentation der planungsrelevanten Daten, mit der Reports, die Managemententscheidungen unterstützen, erzeugt werden können.
V. Freitas et al. führten eine weitere Fallstudie durch, die untersuchte wie die Verwendung von webbasierten Tools, speziell dem hier verwendeten ``VALUE''-Tool, Entscheidungsprozesse unterstützen kann \cite{Value-Based-Decision-Making-Case-Study}. Die Ergebnisse zeigten, dass durch die Verwendung des Tools die Entscheidungsfindung und die Qualität der Entscheidungen durch systematische Herangehensweise verbessert wurde.
% V. Freitas, M. Kemppainen, E, Mendes and P. Rodr ́ıguez, “A systematic literature review of value-based decision-making tools,” Submitted to Information and Software Technology.

\subsubsection{Weitere Tools}
Ein vergleichbares Tool zu dem Konzept, welches in dieser Arbeit entwickelt wird, stellt das ebenfalls webbasierte ``Kanbanize'' dar. Es soll ebenfalls Planungselemente in verschiedenen Ebenen miteinander verknüpfen können. Diese Verknüpfungen werden über sogenannte Workflows realisiert, die Abhängigkeiten abbilden können, und dabei sehr spezifisch konfiguriert werden müssen. Dies lässt komplexere Abhängigkeiten als einfache Verknüpfungen zu, erfordert allerdings komplexere Konfigurationen. ``Kanbanize'' stellt sich dabei als All-In-One-Lösung dar und setzt die ausschließliche Verwendung des Tools auch für die operative Arbeit mit Aufgaben voraus, da es keine Synchronisation mit anderen Tools zulässt. Zusätzlich beschränkt sich ``Kanbanize'' auf die Verwendung von Kanban-Boards. Es erlaubt zudem sehr detaillierte Konfigurationen für das Reporting/Auswerten der aktuell dokumentierten Daten.

\newpage
\section{Konzeption}
\subsection{Prozessdefinition / Anforderungsformulierung}

\subsection{UX-Entwurf für die Abbildung des Prozesses}

\subsection{Datenaggregation}

\newpage
\section{Implementierung}
\subsection{Datenstruktur}
Die Datenstruktur der Anwendung besteht grundsätzlich aus drei abstrakten und fünf konkreten Klassen. Die abstrakten Klassen sind Entitäten, organisationsbasierende Entitäten und verlinkbare Entitäten. Die absoluten Klassen sind Nutzer, Organisation, Level, Gruppe und Aufgabe.

\vspace{20pt}
\begin{center}
    \begin{minipage}{1\linewidth}
        \includegraphics[width=\linewidth]{classDiagramm}
        \captionof{figure}{UML-Diagramm der Datenstruktur}
    \end{minipage}
\end{center}
\vspace{20pt}

Jede absolute Klasse, außer die Nutzer, erben von einer oder mehreren abstrakten Klassen. Entitäten sind allgemeine Objekte innerhalb der Anwendung und stellen die Grundlage der in der Datenbank gespeicherten Datenobjekte dar. Alle Klassen außer Nutzer sind solche Entitäten und implementieren das Interface \verb|IEntity|, welches eine ID zur eindeutigen Identifikation und einen Namen für die Darstellung für den Nutzer beinhaltet. Organisationen und Levels sind direkte Erben dieser Klasse. Die nächste Abstraktionsstufe sind die organisationsbasierenden Entitäten. Diese implementieren zu dem \verb|IEntity| Interface noch \verb|IOrganizationBasedEntity|, welches die ID einer Organisation voraussetzt und die Entität direkt von einer Organisation abhängig macht. Die letzte Abstraktionsstufe sind die verlinkbaren Entitäten. Diese implementieren zu dem \verb|IOrganizationBasedEntity| Interface noch \verb|ILinkableEntity|, welches eine Liste von IDs voraussetzt, mit dem gespeichert wird, mit welchen anderen verlinkbaren Entitäten das Objekt verlinkt ist. Aufgaben und Gruppen sind solche verlinkbare Entitäten.

\subsection{Backend-Architektur}
Das Backend ist eine REST-API, geschrieben mit Node.js und Express in TypeScript und verwendet mongoose als Datenbank-API für MongoDB. Die Architektur beschreibt die den Datenfluss mit drei allgemeinen Komponenten: Router, Controller und Service.
Der Router bestimmt für einen Request welche Funktion eines Controllers aufgerufen wird. Die aufgerufene Controller-Funktion beinhaltet die Business-Logik, die an den Request gebunden ist und führt diese aus. Um Daten aus der Datenbank zu holen oder die geholten Daten zu modifizieren gibt es für jede Datenklasse einen Service, der die benötigten Datenbankoperationen implementiert und somit von der Business-Logik trennt. Für die Interaktion mit der Datenbank muss außerdem ein sogenanntes Model definiert werden, welches die Beschreibung der Klasse also der Type in TypeScript mit der Datenbank-Collection und den Objekten darin verknüpft.

\vspace{20pt}
\begin{center}
    \begin{minipage}{1\linewidth}
        \includegraphics[width=\linewidth]{BE-Struktur}
        \captionof{figure}{Backend Architektur}
    \end{minipage}
\end{center}
\vspace{20pt}

Für die konkrete Kommunikation mit der REST-API werden zu den konkreten Datenobjekten innerhalb der Datenbank zwei weitere Klassen je Objekt-Klasse definiert. Diese Klassen sind sogenannte Data transfer Objects (DTO). DTOs dienen dazu die Kommunikation zu generalisieren und definieren die Daten die der Konsument der API durch einen Request erhalten kann und die Daten, die ein Konsument der API zur verfügung stellen kann, um z.B. ein neues Objekt in der Datenbank zu erstellen. Die zusätzlichen Klassendefinitionen werden durch diese zwei Anwendungsfälle in Read- und Create-/Update-DTOs unterteilt. Wie Create-/Update-DTOs zu einem internen Model gemappt werden und wie aus einem internen Model ein Read-Dto gemappt wird, definiert ebenfalls der zum Model zugehörige Service.

Der Aufbau des Backends gleicht dem Aufbau der Klassen-Abstraktion. Es gibt drei abstrakte Strukturen bestehend aus Router, Controller, Services und Model für jede der drei abstrakten Klassen und fünf absolute Strukturen für jede der fünf absoluten Klassen.


\vspace{20pt}
\begin{center}
    \begin{minipage}{1\linewidth}
        \includegraphics[width=\linewidth]{classDiagramm}
        \captionof{figure}{Abbildung der BE-Struktur}
    \end{minipage}
\end{center}
\vspace{20pt}

\emph{beschreibung de BE-Struktur}

\subsection{Benutzeroberfläche}


\subsection{Visualisierung/Datendarstellung}


\subsection{CI/CD}

\newpage
\section{Evaluation}
\subsection{Praxistest}

\subsection{Optimierungsvorschläge}

\newpage
\section{Fazit}
In dieser Arbeit wurde ein Konzept für eine softwaregestützte Dokumentation von Unternehmensstrukturen für automatisierte Fortschrittsmessung und Werteorientierung entwickelt. Dazu wurde eine Datenstruktur entworfen, die die Abbildung von Unternehmensstrukturen, wie sie für (agiles) Projekt-Portfoliomanagement verwendet werden und eine Datenaggregation für den Fortschritt ermöglicht. Sie soll dabei möglichst flexibel sein, um unabhängig von der konkreten Unternehmensstruktur bzw. dem verwendeten Management-Framework verwendet werden zu können. Die Datenstruktur wurde in einem Prototyp implementiert, der die Funktionalitäten der Datenstruktur in einer Webanwendung umsetzt, um damit das Konzept in einem Praxistest zu evaluieren. Dazu wurde eine fiktive praxisnahe Unternehmensstruktur mit der Unterstützung eines Experten in der Anwendung abgebildet. Dieses Beispiel wurde dann verwendet, um die Anwendung selbst zu testen.

\subsection{Ergebnis}
Die Ergebnisse des Literatur-Reviews zeigten, dass Unternehmen Entscheidungen durch softwaregestützte Dokumentation und automatisiertes Reporting unterstützen sollten, um Entscheidungsfindung zu systematisieren und Entscheidungsqualität zu steigern. Die Ergebnisse des Praxistests zeigten, dass die Datenstruktur dazu in der Lage ist Unternehmensstrukturen abzubilden und die Fortschrittsaggregation eine Form von automatisiertem Reporting darstellt. Des Weiteren sind zwei Kernfunktionalitäten während der Entwicklung des Konzepts identifiziert worden, welche aufgrund der Komplexität nicht implementiert wurden: Die Synchronisation mit operativen Tools wie Jira, anstelle eines einfachen Imports und die Autorisierung und Verwaltung verschiedener Nutzer für Lesebeschränkungen. Außerdem wurden durch den Praxistest drei weitere Funktionalitäten identifiziert, mit denen die Anwendung möglicherweise produktive Anwendung finden kann.

\subsection{Reflexion}
Das Konzept kann Unternehmensstrukturen als solches dokumentieren, sollte aber weitere Möglichkeiten bieten den Kontext von z. B. verschiedenen Ebenen besser zu beschreiben. Die Fortschrittsaggregation ist eine Möglichkeit für eine Unterstützung von wertebasierter Entscheidungsfindung, reicht aber alleine nicht aus, um effektiv die Entscheidungen zu verbessern. Hierzu werden vor allem zeitbasierte Metriken benötigt. Die beiden zuvor erwähnten Funktionalitäten, welche bewusst nicht implementiert wurden, werden definitiv benötigt, um die Anwendung produktiv und Unternehmensweit zu verwenden. Die Visualisierung der Datenstruktur kann ebenfalls verbessert werden, indem Filterfunktionalitäten implementiert werden, welche die Übersicht über größere Strukturen gewährleisten können und den Fokus auf die relevanten Informationen legen. Hierbei sollte insbesondere der iterative Prozess der Planung unterstützt werden. Informationen sollten z. B. nach Iterationen gruppiert und gefiltert werden können.

Für die Berechnung von zeitbasierten Metriken wird in der Datenstruktur eine zeitliche Dimension benötigt, mit der Änderungen über Zeit gespeichert werden können.

Die Datenaggregation findet zurzeit im Frontend, also dem Nutzer-seitigen Teil der Anwendung, statt. Dies könnte insbesondere bei komplexeren Metriken sinnvoller im Server-seitigen Teil der Anwendung stattfinden. Auch die Kalkulation der Metriken wie Velocity o. Ä. könnte vor allem für zukünftig Synchronisierte Projekt aus anderen Tools, wie z. B. Jira bezogen werden, die ohnehin solche Reports zur Verfügung stellen.

\subsection{Ausblick}
Sowohl das entwickelte Konzept, als auch die Implementierung dessen, erlauben ohne weitere Komplikation eine Umsetzung aller identifizierten fehlenden Funktionalitäten und Verbesserungen. Hierbei sollten weitere Metriken, die über die Fortschrittsmessung hinaus gehen, implementiert werden, da es sich bei dieser Arbeit nur um einen Prototyp handelt, der die Aggregation von Daten über mehrere Ebenen hinweg am Beispiel der Fortschrittsmessung demonstriert.
Mit dieser Umsetzung kann das Konzept dann auch realistisch in einem tatsächlichen Praxisumfeld über einen längeren Zeitraum getestet werden, um die Effektivität der Unterstützung von wertebasierter Entscheidungsfindung zu evaluieren.


\newpage
\addcontentsline{toc}{section}{Literaturverzeichnis}
\printbibliography

% \newpage
% \addcontentsline{toc}{section}{Anhang}
\section*{Anhang}


% would be correct but not work
% \begin{appendix} 
%     \subsection{nummero uno}
%     Das ist der Anhang nummero uno
%     \section{nummero dos}
%     Das ist der Anhang nummero dos
% \end{appendix} 

% kinda cheesy but at least works
\addcontentsline{toc}{subsection}{A.1 nummero uno}
\subsection*{nummero uno}
    Das ist der Anhang nummero uno
\addcontentsline{toc}{subsection}{A.2 nummero dos}
\subsection*{nummero dos}
    Das ist der Anhang nummero dos
\newpage
\addcontentsline{toc}{section}{Selbständigkeitserklärung}
\section*{Selbständigkeitserklärung}
Hiermit erkläre ich, Maximilian Oedinger, dass ich die hier vorliegende Arbeit selbst-ständig und ohne unerlaubte Hilfsmittel angefertigt habe. Informationen, die
anderen Werken oder Quellen dem Wortlaut oder dem Sinn nach entnommen sind, habe ich kenntlich gemacht und mit exakter Quellenangabe
versehen. Sätze oder Satzteile, die wörtlich übernommen wurden, wurden
als Zitate gekennzeichnet. Die hier vorliegende Arbeit wurde noch an
keiner anderen Stelle zur Prüfung vorgelegt und weder ganz noch in
Auszügen veröffentlicht. Bis zur Veröffentlichung der Ergebnisse durch den
Prüfungsausschuss werde ich eine Kopie dieser Studienarbeit aufbewahren und wenn nötig zugänglich machen.

\end{document}
